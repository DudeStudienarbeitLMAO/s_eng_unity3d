\documentclass{beamer}
\usepackage{pgfpages}
\setbeameroption{show notes on second screen=right}
\mode<presentation> {
\usetheme{Antibes}
\usecolortheme{structure}
}
\usepackage{graphicx}
\graphicspath{ {pics/} }
\usepackage{booktabs}
\usepackage{tikz}
\usepackage[utf8]{inputenc}
\usepackage[ngerman]{babel} 

\title[Unity3D]{ePortfolio - Unity3D}
\author{Marcel Borrmann}
\institute[DHBW-Ka]
{
Duale Hochschule Baden-Württemberg Karlsruhe \\
\medskip
\textit{vorstand@cduc.su}
}
\date{\today}

\begin{document}

\begin{frame}
\titlepage
\end{frame}

\begin{frame}
\frametitle{Überblick}
\tableofcontents
\note{Erst trockene Fakten, später wird es interessant}
\end{frame}

\section{Unity3D}
\subsection{Was ist das?}
\begin{frame}
\frametitle{Wat?}

\begin{enumerate}
	\item<1->- Cross-Platform Spiele-Engine
	\item<2->- Verfügbar für Windows, Mac OS und GNU/Linux
	\item<3->- Viele Zielplatformen
	\item<4->- Nicht nur 3D!
\end{enumerate}

\note<1-2>{Erst nur osX in 2005 announced, nun für GNU/Linux, Windows, Mac OS}
\note<3>{ Android,Android TV, Facebook Gameroom, Fire OS, Gear VR, Google Cardboard, Google Daydream, HTC Vive, iOS, Linux, macOS, Microsoft Hololens, Nintendo 3DS line, Nintendo Switch,[14]Oculus Rift, PlayStation 4, PlayStation Vita, PlayStation VR, Samsung Smart TV, Tizen, tvOS, Wii, Wii U, Windows, Windows Phone, Windows Store, WebGL, Xbox 360, and Xbox One}
\note<4>{Seit 4.3 gegen Ende 2013 auch 2D Grafik offiziell unterstützt}
\end{frame}

\subsection{Warum Unity?}
\begin{frame}
\frametitle{Y?}
\begin{enumerate}
	\item<1->- Viel Out of the box
	\item<2->- Visueller Workflow
	\item<3->- Rapid iteration
	\item<4->- Anpassbar
	\item<5->- Viele Plattformen
	\item<6->- Components vs. Inheritance
\end{enumerate}
\note<1>{- Physic simulation, normal maps,screen space ambient occlusion, dynamische Schatten...}
\note<1>{- Das haben jedoch viele Engines, was sie nicht haben:}
\note<2>{fast alles in visuellem Editor verankert, keine Build-Chain von Tools nötig}
\note<3>{Schnelle Prototyp-Zyklen durch Bearbeitung direkt imt Editor, auch wenn Spiel läuft sowie IDE workflow}
\note<4>{Editor selbst durch scripting anpassbar}
\note<5>{Begann mit Mac, innerhalb einiger  Monate auch Windows}
\note<5>{iPhone 2008, Android 2010, Spielekonsolen, }
\note<6>{Enemy - mobile enemy oder stationary enemy - mobile shooter oder flying shooter}
\note<6>{Vgl: Mobile enemy: Motion Component, Enemy Component und Mobile Shooter etc}
\end{frame}
\begin{frame}
\frametitle{Y??}
\begin{enumerate}
	\item<1->- Viele Asset- und Dateiformate
	\item<2->- C\#, JavaScript und Boo
	\item<3->- Asset Store
\end{enumerate}
\note<1>{3ds Max, Maya, Softimage, Blender, Cinema4D, Adobe Photoshop}
\note<2>{Js... Boo hat Python inspirierte Syntax, deprecated. c# am meisten genutzt}
\note<3>{Fluch und Segen, Quelle für fast alles. Mehr später}
\end{frame}

\subsection{Warum nicht?}
\begin{frame}
\frametitle{?!}
\begin{enumerate}
	\item<1->- Der Editor
	\item<2->- Kein linking v. externen Libs
\end{enumerate}
\note<1>{Viele Objekte in der Szene - schlechte Übersicht, mäßige Suche}
\note<2>{Muss immer in jedes Projekt kopiert werden, ein mal schreiben, 20 mal anpassen wenn re-used}
\end{frame}

\section{Demo?}
\begin{frame}
\frametitle{Pray}
	{\Huge DEMO? Vielleicht. Mal sehen!}
\end{frame}

\section{Und jetzt?}
\begin{frame}
	\frametitle{Installiere Gentoo}
	\center\includegraphics[scale=0.4]<1>{cde}
	\begin{enumerate}
		\item<2>- Aber ich will doch gar keine Spiele entwickeln, mag aber Geld!
		\item<3->- Ich will Spiele entwickeln aber weiß nicht wie!
	\end{enumerate}
\note<2>{Kein Problem, ein Asset-Flip reicht, mehr dazu gleich}
\note<3>{Unity In Action!\\}
\note<3>{http://gameprogrammingpatterns.com/ -> Game Programming Patterns is a collection of patterns I found in games that make code cleaner, easier to understand, and faster.\\}
\note<3>{Kollaboriere! Spiele entwickeln ist mehr als nur Entwickeln! Game Design, Level Design -- Profitipp: werde Künstler\\}
\end{frame}
\begin{frame}
\frametitle{Der Asset-Flip}
\includegraphics<1>[scale=0.2]{pics/1497357615717}
\note<1>{Dunkle Seite des Asset-Stores und zu offener Spieleplatformen}
\note<1>{Je nach moralischem Kompass Warnung oder Tipp}
\note<2>{Kennt ihr Unturned?}
\includegraphics<2>[scale=0.2]{pics/Unturned2_Outfield}
\note<3>{Relativ Populärer DayZ Clone, F2P}
\includegraphics[scale=0.5]<3>{pics/unturned}
\note<4>{Ja? Nein? Aber kennt ihr auch Day Survival Begins?}
\includegraphics[scale=0.2]<4>{daysurvivalbegins}
\note<5>{Was ist mit Field Z?}
\includegraphics[scale=0.2]<5>{FieldZ}
\note<6>{Nicht schlimm. Hey, Pixel Z auf dem Mobiltelefon!}
\includegraphics[scale=0.7]<6>{PixelZ}
\note<7>{Uncrowded, für 100 CS Keys zu haben, leider vom Store gelöscht}
\includegraphics[scale=0.2]<7>{uncrowded}
\includegraphics[scale=0.2]<8>{1497377825318}
\note<8>{Wie kann das sein? Offensichtlich stimmt was nicht, selbes Spiel}
\includegraphics[scale=0.4]<9>{UnitZ}
\note<9>{\\ Kurze Suche im Asset Store fördert das hier zu Tage}
\note<9>{\\ Since he (Unturned dev) used a free version of Unity the unity devs technically have full rights to use his code in what they want and thus UnitZ was born}
\includegraphics[scale=0.4]<10>{thinking}
\note<10>{Was will ich damit sagen?\\}
\note<10>{Lest die Lizenzen oder beteiligt mich an den Profiten}
\end{frame}
\section{Ende}
\begin{frame}
\frametitle{Schön war's}
Tschö!
\end{frame}


\end{document} 